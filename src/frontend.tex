\documentclass[11pt]{article}
\title{\textbf{{\color{blue}Flex Friday Exploratory Project Proposal}}}
\author{And/Ore Frontend Team}
\date{}
\usepackage{tikz}
\usepackage{gensymb}
\usepackage{physics}
\usepackage{color}
\usepackage{tabto}
\usepackage[normalem]{ulem}
\usepackage{hyperref}
\setlength{\parindent}{0pt}
\begin{document}

\maketitle

\section{{\color{blue}Vision}}

\subsection{{\color{blue}Purpose}}

In the front end of the And/Ore development team, the team is engaged on designing the game to make it more practical, so the players are more likely to enjoy the game.the team is also making the AI in the game work to its maximum potential, as it will make the game, once again, more enjoyable, entertaining, and easier to play, as the players can play around with their own custom AI, with limitless possibilities. The team is doing this because the team believes that it will be fun and will do it by learning how to create AI.

\subsection{{\color{blue}Goal}}

For goals, the front end team plans on making the game have actual textures instead of numbers and letters. the team also plans to make the game more user-friendly, which includes an interface that is more than just black and white. the team also plans on making the AI features of the game work, so a computer and learn to play the game and work well.

\section{{\color{blue}Approach}}


\subsection{{\color{blue}Objectives}}

\begin{itemize}
	\item Create textures for the game.
		\begin{itemize}
			\item The team will develop textures for the game
			\item The team will create animated enemy and player models to make movement more realistic
			\item The team will work together on a list to organise what the team has to update for textures.
		\end{itemize}
	\item Create a more user-friendly game
		\begin{itemize}
			\item The team will create a more user-friendly game.
			\item The team will learn how to create user interfaces and apply their newly minted knowledge to the game.
			\item This will allow the users to enjoy the game to a higher extent.
			\item Creating a tutorial for new players will create a better new player experience.
			\item Make the game's controls comfortable and not two-handed.
		\end{itemize}
	\item Make an AI
		\begin{itemize}
			\item To make an AI, the team firs tneeds to understand how AI "thinks" and how to teach it.
			\item Make the AI easier to work with both developers and users.
		\end{itemize}
\end{itemize}

\subsection{{\color{blue}Team}}

{\large {\color{orange}Joseph Corwin}}

Joseph Corwin has experience in Java, Javascript, C\#, Bash scripting, and client-server communications. He's participated in several programming projects, such as the previously active Horse Magic Studios, a project that was dedicated to creating a video game using GameMaker Studio. Other projects also include a 4-player networked pong clone with active score tracking and basic AI.

{\large {\color{orange}Caleb Marston}}

Caleb Marston is a junior in the project who specializes in mathematics, neural networks and AI. His main goal on the project is to learn what a neural network is and optomize the current one. He has read a substantial amount on learning ai.

{\large {\color{orange}Nate}}

Nate has some programming experience and is a fast learner. He is mostly interested in Machine Learning.


{\large {\color{orange}Sophey}}

Sophey has some experience with coding, and can draw but plans on improving her digital art skills.

{\large {\color{orange}Matt}}

Matt is a programming novice, continuing to learn HTML and Python.

{\large {\color{orange}Alex}}

Alex feels comfortable using photoshop, but still feels that he has room for improvement.

{\large {\color{orange}Kitten}}

Having previously dabbled with programming, Kitten is a fast learner and has experience with digital art.


{\large {\color{orange}Marissa}}

Marrisa can draw, but needs to improve her digital art skills.

{\large {\color{orange}Cuatro}}

A noticible active member of the communities he associates himself with, Cuatro is a team member willing to learn what it takes to get the product shipped.

{\large {\color{orange}Ellis}}

Ellis can do art.

{\large {\color{orange}Ian}}

Ian knows some programming fundamentals, and can do anything assigned to him with the right attitude.


\subsection{{\color{blue}Milestones}}

\begin{itemize}
	\item 10/08/16
		\begin{itemize}
			\item Version 1.4 Release
				\begin{itemize}
					\item Introduced feeding treats to your neural network.
					\item Split up game server for better performance and to introduce the start of multiple playable worlds.
				\end{itemize}
		\end{itemize}
	\item 10/22/16
		\begin{itemize}
			\item Version 1.5 Release
				\begin{itemize}
					\item Bug Fixes
					\item The release of sphere, an admin panel for game moderators
					\item New feature to heal others/buildings.
				\end{itemize}
		\end{itemize}
	\item 11/05/16
		\begin{itemize}
			\item Version 1.6 Release
				\begin{itemize}
					\item Chat Introduced
					\item Bug Fixes
				\end{itemize}
		\end{itemize}
	\item 11/11/16
		\begin{itemize}
			\item Prototype of Isometric Art Viewer meant to replace the text representation of objects present in the game.
		\end{itemize}
	\item 11/19/16
		\begin{itemize}
			\item Version 1.7 Release
				\begin{itemize}
					\item Bug Fixes
				\end{itemize}
		\end{itemize}
\end{itemize}

\section{{\color{blue}Logistics}}

\subsection{{\color{blue}Resources}}

\begin{itemize}
	\item Collaboration
		\begin{itemize}
			\item \href{https://slack.com/}{Slack, Team Messenging App}
			\item \href{https://github.com/}{Github, Social Coding \& Issue Tracker}
		\end{itemize}
	\item Theory
		\begin{itemize}
			\item \href{http://baxter-academy.org/faculty#hlarsson}{Hal}
			\item Rob
			\item Rubber Duck
		\end{itemize}
	\item Learn to Code Resources
		\begin{itemize}
			\item \href{https://www.jetbrains.com/pycharm-edu/}{PyCharm Edu}
			\item \href{https://codehs.com/}{CodeHS}
			\item \href{http://www.codecademy.com/}{\framebox[1.1\width]{code}\underline{c}ademy}
		\end{itemize}
	\item Development Tools
		\begin{itemize}
			\item Windows
			\item Linux
			\item \href{https://atom.io/}{atom}
			\item \href{https://www.jetbrains.com/pycharm/}{PyCharm}
			\item \href{http://www.syntevo.com/smartgit/}{SmartGit}
		\end{itemize}
\end{itemize}

\subsection{{\color{blue}Budget}}

The Backend team of And/Ore is the only team with expenses.

\end{document}